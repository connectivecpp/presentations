\chapter{Example Code for Presentations}
\hypertarget{md__2github_2workspace_2examples_2README}{}\label{md__2github_2workspace_2examples_2README}\index{Example Code for Presentations@{Example Code for Presentations}}
\label{md__2github_2workspace_2examples_2README_autotoc_md0}%
\Hypertarget{md__2github_2workspace_2examples_2README_autotoc_md0}%
 



This directory contains example source code from the presentations in the \href{../presentations}{\texttt{ presentations}} directory.

All of the example source code has \href{https://github.com/catchorg/Catch2}{\texttt{ Catch2}} unit test code. The Catch2 unit test presentation (of course) features the unit test code. Each directory corresponding to a presentation has a top level CMake file.

Each CMake configure / generate will download and build the Catch2 library as appropriate using the \href{https://github.com/cpm-cmake/CPM.cmake}{\texttt{ CPM.\+cmake}} dependency manager. If Catch2 (v3 or greater) is already installed using a different package manager (such as Conan or vcpkg), the {\ttfamily CPM\+\_\+\+USE\+\_\+\+LOCAL\+\_\+\+PACKAGES} variable can be set which results in {\ttfamily find\+\_\+package} being attempted. Note that v3 (or later) of Catch2 is required, which results in faster unit test build times (once the initial Catch2 library is compiled) due to Catch2 changing from a "{}header only"{} library to a library with compiled objects (in addition to the header files).

All of the example code builds and successfully runs (as unit tests) on Ubuntu (Linux), Windows, and mac\+OS, using Git\+Hub Actions (the Git\+Hub continuous integration facilities). Specifically, {\ttfamily ubuntu-\/latest}, {\ttfamily windows-\/latest}, and {\ttfamily macos-\/latest} or {\ttfamily macos-\/14} are the target runners specified in the build matrix in the workflow YAML file.

All of the CMake files have C++ 20 set as the required C++ standard, but only small portions require C++ 20, so most of the code can be used with older C++ standards.

The {\ttfamily intro\+\_\+generic\+\_\+programming} example uses a third party {\ttfamily decimal} library from \href{https://github.com/TimQuelch/decimal}{\texttt{ Tim Quelch}}. The CMake configure / generate step requires the {\ttfamily decimal} test code to be bypassed in the build (it uses an older version of Catch2) -\/ see notes below for specifics.

To build and run (all of) the example test programs\+:

First clone the {\ttfamily presentations} repository, then create a build directory in parallel to the presentations directory (this is called "{}out of source"{} builds), then {\ttfamily cd} (change directory) into the build directory. The CMake commands\+:


\begin{DoxyCode}{0}
\DoxyCodeLine{cmake\ -\/D\ DECIMAL\_ENABLE\_TESTING:BOOL=OFF\ ../presentations/examples}
\DoxyCodeLine{}
\DoxyCodeLine{cmake\ -\/-\/build\ .}
\DoxyCodeLine{}
\DoxyCodeLine{ctest}

\end{DoxyCode}


For additional test output, each test can be invoked individually, for example\+:


\begin{DoxyCode}{0}
\DoxyCodeLine{std\_span/std\_span\_test\ -\/s}

\end{DoxyCode}
 